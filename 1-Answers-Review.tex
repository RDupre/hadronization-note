\documentclass[12pt]{article}                                            
\usepackage {epsfig}
\textwidth  16.0cm
\textheight 21.5cm
\topmargin -1.0 cm
\oddsidemargin 0.5cm
\evensidemargin 0.5cm
\pagestyle{plain}
\headheight 0.3cm 
\usepackage[dvipsnames]{color} 

%%%%%%%%%%%%%%%%%%%%%%%%%%%%%%%%%%%%%%%%%%%%%%%%
\begin{document}
%%%%%%%%%%%%%%%%%%%%%%%%%%%%%%%%%%%%%%%%%%%%%%%%
\definecolor{Myred}{named}{Red}
\pagestyle{plain}
\begin{titlepage}
\begin{flushright}
~~~~~~~~~~~~~~~~~~~~~~~~~~~~~~~~~~~~~~~~~~~~~~~~~~~~~
~~~~~~~~~~~~~~~~~~~~~~~~~~~~~~~~~~~~~~~~~~~~~~~~~~~{\today}
\end{flushright}
\vskip 3.cm
\centerline{\bf \color{Myred}{M. Wood (chair), S. Manly, and A. Kim} }
\vskip 0.5cm
\centerline{\bf \LARGE \color{Myred}{Analysis Review of the :} }
\vskip 1.cm
\centerline{\bf \Large ``Study of the Hadronization}
\vskip 0.5cm
\centerline{\bf \Large of Charged Pions”''}
\vskip 0.5cm
\centerline {\bf  by R. Dupre et al.}
\vskip .5cm
%%%%%%%%%%%%%%%%%%%%%%%%%%%%%%%%%%%%%%%%%%%%%%%%%%%%%%%%%%%%%%%%%%
\end{titlepage}
\setcounter{page}{2}

The committee has reviewed the analysis note and has a number of comments that should be 
addressed.  It has been a long time since the data were taken, and it is an interesting study, so 
we encourage the authors to move forward with the improvements in order to bring it to 
publication.

This report is organized in 3 sections: lager issues to resolve, questions from the reviewers, and 
typographical corrections.  There may be some repetition between the 3 sections.  The notation 
is P\# for page \#.

\section{Larger Issues }
1.
Ownership of the pi+ channel.  This was brought up in an email from the committee chair 
(M. Wood).  Before this analysis can move on to round 2 of the review, the ownership issue 
has to be resolved.  The NPWG chair (also M. Wood) has left it up to the eg2 group to 
resolve it.



2.
The particle ID and cuts seem to be well understood.  The acceptance correction and 
radiative corrections need to be developed.  There is confusion about how the acceptance 
corrections were done (see Section 2 below).  The radiative corrections are incomplete as 
reported in the note.  Both corrections need to be well-understood before the committee can 
approve of the analysis.



3.
In the next version, line numbers need to be added to the document for easier reference.

\section{Reviewers’ Questions/Comments}
1.
P3, para 2- The discussion about the importance of this to neutrino experiments misses the 
main point.  In neutrino experiments, it is true that nuclei other than hydrogen are used to 
increase the event rate and to avoid explosions.  That means that nuclear effects come into 
play.  Fermi motion and final state interactions and correlations cause confusion in the 
extraction of the signal and reconstruction of the energy of the incoming neutrino.  This 
leads to less precise determination of the oscillation parameters.  The modeling of nuclear 
effects is necessary to understand and correct for these effects.  Pion production 
measurements like this provide ways to tune the models and look at the A dependence.  The 
A dependence is very important in part because in the long baseline experiments sometimes 
the near detector and far detectors are made of different materials.  In DUNE, for example, 
the far detector will be a liquid argon TPC.  It is not yet clear that a liquid argon TPC will 
work well as a near detector in such an intense neutrino beam.  Clearly you don’t want to 
say all this on page 3 of the paper.  But I think an additional sentence is warranted.




2.
P10 footnote: "single photon" could be confusing in this context, since the mean number of 
10 is from multiple optical photons (produced from a single electron).



3.
P13 - The number 0.03 - is it from the fit results of previous analysis note you mentioned in 
the first chapter? Or is it eyeball fit? If former you could add 1D histogram with fit results 
plotted to Figure 9?



4.
P16 - similarly, 1D fits for positive pion $\Delta\beta$?



5.
P16 - You state that the target position could vary from run to run, but “did not happen too 
often”.  Can you quantify this?  



6.
P17, Fig 12 - what solid target is shown?  Are the shifts the same for electrons and pions?



7.
P18, Start of section 2.2.5 - I’m not sure I understand the logic in this.  If you have a drift 
chamber problem, it will be manifest in both the solid and liquid target data and the ratio of 
these should be relatively stable.  Why do you not determine run quality by looking at the 
event rate per coulomb of beam current?  For this quantity, a change in the detector 
operation will show up.



8.
P19, FIg. 14- Hard to tell how much data was discarded.  Maybe give a table of means and 
sigmas for each target or show the cuts on the figure are horizontal lines.



9.
P19 - If you use "c" constant in dimensions of Q2, then use it for W as well.



10.
P19, Sec. 2.3 - Show plots of Q$^2$, W and y with the cuts.



11.
P19, eq. 7 - Is it really just the sum of weights or are they the counts multiplied by the 
weights?



12.
P23, I’d be very worried about the MC modeling acceptance on falling edges like that at the 
right of the top right plot.



13.
P25, I find this discussion quite confusing.   At top you talk about two binnings.  It’s not just 
binning that is changing, right?  You are using different variables.  Then you talk about the 
extracted weight, but that weight is defined with only one of those sets of variables.  You 
also do it for the other set?  Then you an arbitrary weight cut to take care of the problems 
listed in the middle of the page?  Why not just increase the size of the bins in regions where 
you have too few events or large bin migration?  I don’t see how the procedure you use 
does not lead to holes in your data somehow.  And I don’t really understand what you are 
doing with the second weighting factor.  How is this not doing a double correction for 
acceptance, i.e., a double counting?  This whole section needs to be reworked.  I am fairly 
certain that I don’t really understand what you are doing but it leaves me worried that 
something is amiss.  Anyway, that’s not where you want readers to be.  



14.
P26, bottom, if you have regions with very low acceptance, should they be cut out?



15.
P26 - "consistent with HERMES results" - could you add HERMES points to the plot with 
your data points to show what you mean as consistent?



16.
P27 fig.22-
 It's unclear to which point the error bar is associated. I think the common 
practice is to slightly shift markers horizontally.



17.
P27, Fig. 22 - the right column of plots is not the same as those in Fig. 15.  Why change the 
variables?



18.
P28 - What is Coulolmb Correction applied to? Momentum of charged particles? Could you 
plot effect of this correction on particles? Is it really big enough to be visible within our 
resolutions ?



19.
P30 - 
fig.24 is missing X axis label. You also may consider 2D histogram for better display.



20.
P31 - I don’t understand the isospin correction.  It does smooth out the data in Fig. 27; 
however, I am missing the physics motivation.  Why does it matter if the quark came from a 
proton or a neutron?  I think you just clarify the explanation.



21.
P33, Sec. 2.5.1 - list the negligible effects.



22.
P34, Fig. 28 - why is the resolution poorer than in Fig. 12?



23.
P34, why do you have a 1\% normalization error in all the multiplicity ratios when you say just 
above that for this particular sample with two particles in the final state the events in the test 
regions dropped to 0.01?  Doesn’t that mean the target contamination normalization error is 
smaller than 1\%?



24.
P34, with this technique of comparing results using the two different sets of variables (you 
call it two binnings, but I think that is sort of misleading), you are using the same events.  
Does this not mean you have a statistical contribution to this systematic error?  You say you 
noticed larger systematic errors for the pi- events due to lower acceptance and bigger 
weights.  Might it also be lower statistics and so you have statistical fluctuations contributing 
to the systematic error?  If so, you are sort of double-counting the statistical error and 
inflating the systematic error more than it should be.  



25.
P35 - It would clarify the discussion with a summary or table of the point-to-point systematic 
errors.



26.
P35 -
 add plots with systematic uncertainty bands on it to demonstrate bin-by-bin values.



27.
P35, end of top paragraph, you say you don’t include the acceptance systematic, but it looks 
to me like it is included in Table 8?  What am I missing?



28.
P36, top paragraph, “that a factorization holds” means what??  Should explain.  



29.
P36 top of section 3.1.1, 5% versus 5% looks more like 1 sigma rather than the 1.5 sigma 
you say in the text.



30.
P36, second paragraph of section 3.1.1, “... might be modified if the production time occurs 
inside the nucleus.”  I think I know what you mean but the language makes no sense.  You 
mean if the formation takes place inside the nucleus, right?



31.
P42, top, “... that is seen in our result.”  Where do I see what you are discussing in the top 
two lines of this page.  I could not find whatever it is.  What should I be looking at?

\section{Typographical Corrections}

Abstract L2, “... non-perturbative; therefore, only...”

P3, para 2:  “... models, including ones relevant for RHIC, because the size of the cold nuclei 
used in the JLab experiments is stable and known.”

P4, L1, “... the study of pion production over ...”

P4, L3, “... section 1.3, we provide a...”

P5, para 3, L3, “This kind of model is suitable...”

P6, L4, “... a suppression of hadron production ...and it can even reverse sign and increase at 
very low z.”

P6, bottom paragraph, line 3, “... excess in delta(pt)\^2.  It is notable that ... behavior in models 
... models do not yet treat baryon hadronization.”

P9, L6, “filled with liquid deuterium”

P9, para2, end of para, “... and detector properties cancel in the nuclear ratio.”

P10, L2, okay for a technical note but here the sentence should be rephrased to something like 
“For pions, we require a signal in the drift chambers (DC) and scintillator counters (SC) in 
addition to the successful reconstruction of the time-based tracking.”

P10, L10, “The derivation of these quantities is presented in section 2.3, while a list of 
corrections  is summarized in section 2.4.  The systematic uncertainty budget is detailed in 
section 2.5, and ...”

P10, just below equation 3, “ ... and the parameter b varies with momentum, as ...”

P13, L1 of section 2.2.2, sentence makes no sense as written. Do you mean something like, 
“Negatively charged pion candidates must not pass the electron selection criteria.”

P15, L2 of section 2.2.3 “... (see figure 11) due to a significant...”

P15 bottom, “These cuts minimize the kaon...”

P18, 6 lines from the bottom - “As Figure 14 ...”

P20, L1 of section 2.3.2 “... in figure 15 a few ...”

P21, L2, “... processed by CLAS software (GSIM, GPP and user\_ana) [what do these mean to a 
non-CLAS reader? supply references] to simulate the detector and the reconstruction process in 
a fashion similar to that done with the experimental data.”
Start of next paragraph, “The simulated data are processed like the experimental data by 
applying ...”  “ ... quite well the detector response, yet ...”

P22, bottom, “... space of our two-particle final state.  However, to evaluate the systematic 
errors associated with ...”

P33, 2 lines from bottom, “re-scattering on detector material”

P35, top of section 2.5.4, “... contributions from particle misidentification and acceptance ... 
effects, target misidentification and the isospin ...”

P36, second paragraph of section 3.1.1 -
 change conciliate to reconcile

P38, Para 3, L2, change signification to significance

P38, bottom, “However, our results (figure 32(left)) show no dependence besides ... and what is 
left will be washed ...”

P41, first paragraph, what does HERMES say about Q2 dependence?  Line 4, “... in figure 34, 
does not indicate ...”

P41, bottom paragraph, “... wide A coverage gives a outright indication ... found to be much 
smaller than that seen by HERMES [15], which ....from BDMPS [29] correlate pt\^2 with the 
nuclear radius.”

P42, top paragraph, last line, “colored parton does not interact with the whole nucleus which 
limits the nuclear effect.”

P42 second paragraph, L2, “they predict a rise of delta pt\^2 with Q\^2”
Same paragraph, L4, “... 38) gives a similar result.”
\end{document}
