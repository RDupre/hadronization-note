
\section{Introduction}
\label{physics}

The hadronization process by which parton fragment into hadrons is 
experimentally well studied and fragmentation functions fitted from 
data are measured on a large kinematic range~\cite{Albino:2008fy}.
Nevertheless, the hadron formation is a non perturbative process and can not 
be theoritically described, only a global picture 
(see figure \ref{fig:hadro}) can be drawn for
energies above the resonance region. In deep inelastic scattering the
virtual photon interact with a quark which propagates quasi-free emitting
gluons during the 
so-called production time. After neutralizing its color, the quark becomes 
a pre-hadron which will eventualy form a hadron after the formation time.
On nuclear target the process of hadronization occurs, at least partialy, in 
nuclear matter, then one can extract the production time using the different 
behaviour of quarks and colorless pre-hadrons in medium. The measurement of
pions is especially suited to lead that kind of analysis because they are easy
to detect and abundant in deep inelastic scattering.

\begin{figure}[htbp]
\centering
\includegraphics[width=10cm] {fig/hadro.png} 
\caption {Sketch of the hadronization process.}
\label{fig:hadro}
\end{figure}

The hadronization in nuclear matter is a key process in QCD 
because it leads to major theoritical uncertainties in numerous measurements.
In Relativistic Heavy Ion Collisions (RHIC), the 
measured hadrons are produced in hot and evolving 
medium, the models describing those collisions can be tested in deep inelatic scattering 
where the size of the medium is stable and known. Quantitative
understanding of the hadronization in nuclei is also of interest for
neutrino experiments which often use nuclei to maximize their cross sections.
The hadronization can give important information about the 
nuclei through quark energy loss, for exemple \cite{Arleo:2003yf} link 
energy loss with gluon density and \cite{Kopeliovich:2010aa} link the $\Delta \langle
P_T^2 \rangle$ with the saturation scale. Other models \cite{Gallmeister:2007an}
can give information about how the pre-hadron evolves into hadron but are based
on very different assomptions, therefore it is very important to discriminate between
models. 

In order to constrain the existing models we extract our results
using a tight multi-dimensional binning. 
The importance of the EG2 data comes from its high statistics which permit, 
this multi-dimensional measurement of both pions on a 
large kinematic range.

We present in section \ref{sec:theo} the different familly of models describing
hadronization in nuclear matter then in section \ref{sec:exp} we make a rapid
overview of the published data on the topic. For more detailed information we 
refer the reader to the review from Accardi {\it et al.} \cite{Accardi:2009qv}.


\subsection{Kinematic Variables and Observables}

In this analysis we use the following semi-inclusive Deep Inelastic Scattering (DIS) variables:
\begin{itemize}
 \item $\nu = E_i - E_f$ is the energy transfered by the lepton probe in the laboratory frame, $E_i$ is the beam energy and $E_f$ is the scattered electron energy,
 \item $Q^2 = 4 E_i E_f \sin ^2(\theta_e / 2)$ is the 4-momentum transfered and $\theta_e$ is the scattered electron angle,
 \item $x_{Bj} = {{Q^2} \over {2 M_n \nu}}$ is the fraction of the proton momentum carried by the struck quark and $M_n$ is the nucleon mass,
 \item $W^2 = M_n^2 - Q^2 + 2 M_n \nu$ is the mass squared of the hadronic system produced,
 \item $z_h = E_h / \nu$ is the fraction of energy available carried by the measured hadron and $E_h$ is the hadron energy,
 \item $P_T^2$ is the transverse momentum with regard to the virtual photon,
 \item $\phi_h$ is the angle between the leptonic plan (containing the initial and scattered electrons) and the hadronic plan (containing the virtual photon and hadron),
 \item $x_F = P_L/P_L^{max}$ is the longitudinal momentum fraction carried by the hadron calculated in respect to the virtual photon direction in the laboratory frame.
\end{itemize}
%TODO Add t

We define the multiplicity ratio as:
\begin{equation}
R_A^h (Q^2,\nu,z_h,P_T^2) = {{N_A^h (Q^2,\nu,z_h,P_T^2) / N_A^e (Q^2,\nu)} 
                       \over {N_D^h (Q^2,\nu,z_h,P_T^2) / N_D^e (Q^2,\nu)}},
\end{equation}
$N^e_A$ is the number of electrons measured from a target $A$ and $N_A^h$ is the number
of semi-inclusive hadrons $h$ (detection of both an electron and a hadron is required) from the
target~$A$. The multiplicity ratio represents the attenuation of the hadron $h$ in a 
nuclear target~$A$.
\newline

We define the transverse momentum broadening as:
\begin{equation}
\Delta \langle P_T^2 \rangle = \langle P_T^2 \rangle_A - \langle P_T^2 \rangle_D,
\end{equation}
$\langle P_T^2 \rangle_A$ is the mean transverse momentum measured in a target $A$.


\subsection{Theoretical Efforts}
\label{sec:theo}

The different models explaining the data can be separated in three families,
some assume that the quark looses energy in the medium and that either, hadronization occurs
outside the medium or the hadronic interaction is 
negligible, others neglect the quark energy loss and consider
only the hadron and pre-hadron absorption, and finally some models consider both
interactions. We will give an example of each case to illustrate. For more
detailed information, we refer again to~\cite{Accardi:2009qv}, which is more exaustive and
highlights also models used for RHIC experiments.

In \cite{Wang:2002ri}, E.~Wang and X.-N.~Wang, describe HERMES data using only 
the energy 
loss from the parton, the suppression observed is then due to the fact that a 
lower energy quark will fragment into a lower number of hadrons and a lower 
$z$. That kind of model is used in both RHIC and nuclear DIS and permit a commom 
interpretation of hadron suppression in nuclear matter. Lot of different
calculations of the energy loss exist, the main parameter for those is $\hat q$
(GeV$^2$ fm$^{-1}$) which is the transverse momentum per unit of length of a 
quark after propagating in the nuclear matter. Its value is related to the $P_T$ 
broadening observable. The main difficulty for quark energy loss
models is the lack of a coherent description for both multiplicity ratios and $P_T$ 
broadening. Recent models used a $\hat q$ to reproduce multiplicity ratio
which is to large to describe the transverse momentum broadening.
% TODO Add a ref

The GiBUU model \cite{Gallmeister:2007an} is a transport model based on the 
Boltzmann equation
using hadronic and pre-hadronic interactions in the nuclear matter, with no
quark energy loss involved. This model reproduces very well most of hadrons'
multiplicity ratios, 
however the $\Delta \langle P_T^2 \rangle$ variable is not described at all by 
these hadron absorption models.

B.Z.~Kopeliovich \cite{Kopeliovich:2008uy} described the process by neglegting 
neither the quark 
energy loss nor the hadron absorption, and by using transverse momentum and
multiplicity ratios to differenciate their effects. In that case the transverse 
momentum broadening is linked to the quark energy loss and the multiplicity ratio
suppression is explained by the hadron absorption in the nuclear medium.

To conclude it is important to point out that no consensus is reached on which 
mecanisms are dominant. It is therefore important to perform precise measurements with
the right observables to solve this problem.


\subsection{Previous Mesurements}
\label{sec:exp}

Hadron multiplicity ratios in nuclei were measured in numerous lepton 
facilities: L.S.~Osborne {\it et al.} \cite{Osborne:1978ai} at SLAC,
L.~Hand {\it et al.} \cite{Hand:1978tx} and the E665 collaboration \cite{Adams:1994ri} at FNAL
and the European Muon Collaboration \cite{Arvidson:1984fz,Ashman:1991cx} at CERN. These 
measurements revealed the general features of the hadronization mechanism
in nuclei: we observe a suppression hadron production in heavier nuclei.
This suppression appears to be reduced at higher $\nu$ and lower $z$, and it 
can even inverse and become an increased yield at very low $z$. 

The figures \ref{fig:her1}, \ref{fig:her2} and \ref{fig:her3} show a sample of 
the most recent data from the HERMES collaboration~\cite{Airapetian:2007vu}, 
where numerous 
hadrons were studied individually and new variables linked with transverse 
momentum were used in addition of the usual multiplicity ratio.

\begin{figure}[htbp]
\centering
\includegraphics[width=14cm] {fig/Hermes/pi0hermes.png} 
\caption {Multiplicity ratio of $\pi^0$ as a function of various kinetic 
variables from HERMES collaboration \cite{Airapetian:2003mi}}
\label{fig:her1}
\end{figure}

\begin{figure}[htbp]
\centering
\includegraphics[width=14cm] {fig/Hermes/hermes1.png} 
\caption {Multiplicity ratio of $\pi^+$, K$^+$, protons and anti-protons as a 
function of various kinetic variables from HERMES collaboration \cite{Airapetian:2007vu}}
\label{fig:her2}
\end{figure}

\begin{figure}[htbp]
\centering
\includegraphics[width=7cm] {fig/Hermes/pthermes.png} 
\caption {Transverse momentum broadening for various particles
from HERMES collaboration \cite{Airapetian:2009jy}}
\label{fig:her3}
\end{figure}

The HERMES data, because of their precision,
permit to have an insight into new features. One can cite the behavior of K$^+$s
which have less attenuation than pions (see figure \ref{fig:her1}) but more 
$\Delta \langle P_T^2 \rangle$. It is difficult to reproduce this in 
models where only one stage of hadronization is taken into account to 
explain all effects.
Also the different behavior of protons compare to anti-protons
(see figure \ref{fig:her2}) is interesting
and need a precise analysis since most of the theoretical models do not treat 
baryon hadronization yet.

